
%\documentclass{sig-alternate}

% English flaws:
% "can be made ON ..." and not "can be made upon"
% "disadvantage OF disruptING" and not " disadvantage to disrupt"
% "requesting something" and not "requesting for something"


%\documentclass[runningheads]{llncs}

%\documentclass[times, 10pt,twocolumn]{article} 
%\documentclass[preprint,10pt]{sigplanconf}

\documentclass[runningheads]{llncs}
%\documentclass{sig-alternate}


%\usepackage{latex8}
%\usepackage{times}

%\documentstyle{llncs}
%\documentclass[preprint,10pt]{sigplanconf}
% constants
\newcommand{\Title}{MrHolmes: Beyond Classical Continuous Integration}
\newcommand{\TitleShort}{\Title}
\newcommand{\Authors}{Ivan Rojas~~~~~Alexandre Bergel}
\newcommand{\AuthorsShort}{I. Rojas, A. Bergel}



\usepackage{xspace}
\usepackage{ifthen}
\usepackage{amsbsy}
\usepackage{amssymb}
\usepackage{balance}
\usepackage{booktabs}
\usepackage{graphicx}
\usepackage{multirow}
\usepackage{needspace}
\usepackage{microtype}
\usepackage{bold-extra}

% references
\usepackage[colorlinks]{hyperref}
\usepackage[all]{hypcap}
\setcounter{tocdepth}{2}
\hypersetup{
	colorlinks=true,
	urlcolor=black,
	linkcolor=black,
	citecolor=black,
	plainpages=false,
	bookmarksopen=true,
	pdfauthor={\Authors},
	pdftitle={\Title}}

%
%\usepackage{amssymb}
%\usepackage{graphicx}
%%\usepackage{amsthm}
%\usepackage{ifthen}
%\usepackage{xspace}
%\usepackage{alltt}
%%\usepackage{psfig} 
%\usepackage{latexsym}
%\usepackage[latin1]{inputenc}
%%\usepackage{psfig}
%\usepackage{url}            
%%\usepackage{amssymb}
%%\usepackage{amsfonts}
%%\usepackage{amsmath}
%\usepackage{stmaryrd}
%\usepackage{enumerate}
%\usepackage{cite}
%\usepackage[pdftex,colorlinks=true,pdfstartview=FitV,linkcolor=blue,citecolor=blue,urlcolor=blue]{hyperref}
%\usepackage{xspace}
%%%%%%%%%%%%%%%%%%%%%%%%%%%%%%%%%%%%%%%%%%%%%%%%%%%%%%%%%%%%

\newcommand{\myhref}[1]{\href{http://#1}{#1}}

%%%%%%%%%%%%%%%%%%%
% Please remove this for the camera ready
%%%%%%%%%%%%%%%%%%%
\pagenumbering{arabic}
\pagestyle{plain}
%%%%%%%%%%%%%%%%%%%
%%%%%%%%%%%%%%%%%%%%%%%%%
%:Generic Macros
\newboolean{showcomments}
\setboolean{showcomments}{false}
\ifthenelse{\boolean{showcomments}}
  {\newcommand{\bnote}[2]{
	\fbox{\bfseries\sffamily\scriptsize#1}
    {\sf\small$>$\textit{#2}$<$}
    % \marginpar{\fbox{\bfseries\sffamily#1}}
   }
   \newcommand{\cvsversion}{\emph{\scriptsize$-$Id: macros.tex,v 1.1.1.1 2007/02/28 13:43:36 bergel Exp $-$}}
  }
  {\newcommand{\bnote}[2]{}
   \newcommand{\cvsversion}{}
  } 

\newcommand{\etal}{\emph{et al.}\xspace}
\newcommand{\here}{\mynote{***}{CONTINUE HERE}}
\newcommand\nb[1]{\mynote{NB}{#1}}
\newcommand\fix[1]{\mynote{FIX}{#1}}
\newcommand\ab[1]{\bnote{Alex}{#1}}
\newcommand\rr[1]{\bnote{Romain}{#1}}
\newcommand\tk[1]{\bnote{Tobias}{#1}}
\newcommand\fb[1]{\bnote{Felipe}{#1}}
\newcommand\lr[1]{\bnote{Lukas}{#1}}
\newcommand\ml[1]{\bnote{Mircea}{#1}}


\newcommand{\sep}{\texttt{>>}\xspace}

\newcommand{\stBar}{$\mid$}
\newcommand{\ret}{\^{}}
\newcommand{\myparagraph}[1]{\noindent\textbf{#1.}}
\newcommand{\eg}{e.g.,\xspace}
\newcommand{\ie}{i.e.,\xspace}
\newcommand{\co}[1]{{\small\textsf{#1}\xspace}}
\newcommand{\secref}[1]{Section \ref{sec:#1}}
\newcommand{\seclabel}[1]{\label{sec:#1}}
\newenvironment{tcode}
    {\footnotesize\sf\begin{tabbing} xx\=xxxxxx\=xxxxxx\=xxxxxx\=xxxxxx\=xxxxxx\=xxxxxx\=xxxxxx\=xxxxxx\=xxxxxx\=xxxxxx\=xxxxxx\=xxxxxx\=\kill}
%xx\=xx\=xx\=xx\=xx\=xx\=xx\=xx\=xx\=xx\=\kill}
    {\end{tabbing}\sf\normalsize}

\newcommand{\figref}[1]{Figure~\ref{fig:#1}}
\newcommand{\figlabel}[1]{\label{fig:#1}}
\newcommand{\tablabel}[1]{\label{tab:#1}}
\newcommand{\tabref}[1]{Table~\ref{tab:#1}}

\usepackage[normalem]{ulem} % for \sout
\usepackage{xcolor}
\usepackage{color}
\newcommand{\ra}{$\rightarrow$}
\newcommand{\ugh}[1]{\textcolor{red}{\uwave{#1}}} % please rephrase
\newcommand{\ins}[1]{\textcolor{blue}{\uline{#1}}} % please insert
\newcommand{\del}[1]{\textcolor{red}{\sout{#1}}} % please delete
\newcommand{\chg}[2]{\textcolor{red}{\sout{#1}}{\ra}\textcolor{blue}{\uline{#2}}} % please change

%%%%%%%%%%%%%%%%%%%%%%%%%
%:Specialized macros

%%%%%%%%%%%%%%%%%%%%%%%%%

\begin{document}
%SPRINGER
%\title{Reconciling Static Type Declaration and Dynamic Scripting Languages}
%\author{Alexandre Bergel}
%\authorrunning{A. Bergel}
%\institute{ADAM Project, INRIA Futurs, Lille, France\\
%\co{\href{http://www.bergel.eu}{www.bergel.eu}}\\
%}

%ACM
%\conferenceinfo{ECOOP}{'11}

\title{\Title}
\titlerunning{\Title}

%\author{Alexandre Bergel\\
%Pleiad Lab, Department of Computer Science (DCC),\\ University of Chile, Santiago, Chile\\ [1 ex]
%\href{http://bergel.eu}{http://bergel.eu}
%}

\author{\Authors}
\authorrunning{\AuthorsShort}

\institute{PLEIAD Lab, Department of Computer Science (DCC), \\University of Chile, Santiago, Chile\\
\url{ivanrohe@gmail.com}~~~~
\url{http://bergel.eu}
}


\newcommand{\spp}{~~~~~~~}

\maketitle

%\cvsversion

%\begin{center}
%\textbf{Accepted at ECOOP'11 - Do not distribute, this is not the final version}
%\end{center}

\begin{abstract}
% What is the problem


% Why is the problem a problem?
% What's the surprising Idea?
% What's the consequence?
\end{abstract}

%\category{D.3.3}{Programming Languages}{Language Constructs and Features}
%\category{D.1.5}{Programming Languages}{Object-oriented Programming}
%\terms{Language, Design}
%\keywords{Multi-language system, interoperability, SmalltalkLite, JavaLite, dynamic languages}


%\begin{figure}[!]
%\begin{center}
%\includegraphics[scale=0.68]{figures/controlleddisruption}
%\caption{The method \co{senseAndSend} is cut down into small pieces, called fragments.} \figlabel{controlleddisruption}
%\end{center}
%\end{figure}

%:%%%%%%%%%%%%%%%%%%%%%%%%%%%%%%%%%%%%%%%%%%%%%%%%%%
%\emph{Note for the proceeding reader: this paper makes use of colors. Although not mandatory for its understanding, an online (colored) version of this paper will ease the reading.}

\section{Introduction} \seclabel{introduction}

% What is the problem
Exercising a continuous process of quality control has recently become a popular trend in many software engineering teams, both in open-source and in industry. 

% Why is the problem a problem?
% What's the surprising Idea?
% What's the consequence?

%:%%%%%%%%%%%%%%%%%%%%%%%%%%%%%%%%%%%%%%%%%%%%%%%%%%
\section{Continuous Integration} \seclabel{problem}

\paragraph{Continuous integration.} The process related to a continuous quality control and monitoring is commonly referred as \emph{continuous integration}. 

\paragraph{Limitations.} 

\begin{itemize}
\item closed history model: new metrics cannot be easily added
\item global metric application: the history model is the same for all the projects
\end{itemize}

%:%%%%%%%%%%%%%%%%%%%%%%%%%%%%%%%%%%%%%%%%%%%%%%%%%%
\section{MrHolmes} \seclabel{klotz}

%=======
\subsection{MrHolmes in a nutshell} \seclabel{nutshell}

MrHolmes is a test server created by de University of Chile.

Its main function is to test different projects built with smalltalk (Pharo, Seaside, among others).

Also, It shows information about tested project, like number of lines, clases or another measure, and others that are specific for the project.

%=======
\subsection{Adding new metrics}

If to want to add a new project, you have to put the project name, but the project must have 
the the following characteristics.

\begin{itemize}
\item To be a squeaksource project. 
\item To have a configuration package.
\end{itemize}

%=======
\subsection{Adding new rendering}

Each project in MrHolmes contents a list of result for each day. That result is represented by a sign and it meaning the project state.

The signs are:

Icon glossary of project results
\begin{itemize}
\item The result was successful, and does not have any error or warning test.
\item The result has warning tests, but does not have any error test.
\item The result has error tests.
\item The project execution has failed, because a error has ocurred during the loading of project configuration.
\end{itemize}

Each result shows the following datas:
\begin{itemize}
\item Successful tests.
\item Warning tests.
\item Error tests.
\item Benchmarks (aditional information and it depends of the project).
\item Error loading (It will show the error description if a error happen during the project load).
\end{itemize}

%=======
\subsection{Metamodel} \seclabel{metamodel}


%:%%%%%%%%%%%%%%%%%%%%%%%%%%%%%%%%%%%%%%%%%%%%%%%%%%
\section{Applications} \seclabel{applications}

%=======
\subsection{Performance monitoring}

%=======
\subsection{...}


%:%%%%%%%%%%%%%%%%%%%%%%%%%%%%%%%%%%%%%%%%%%%%%%%%%%
\section{Implementation} \seclabel{implementation}

\paragraph{Seaside front-end}

\paragraph{Cron}

\paragraph{Managing errors}

%:%%%%%%%%%%%%%%%%%%%%%%%%%%%%%%%%%%%%%%%%%%%%%%%%%%
\section{Related Work} \seclabel{relatedwork}

\paragraph{Hudson}

\paragraph{Jenkins}

%:%%%%%%%%%%%%%%%%%%%%%%%%%%%%%%%%%%%%%%%%%%%%%%%%%%
\section{Conclusion} \seclabel{conclusion}



%{\small 
%\paragraph{Acknowledgment.} We thank Mircea Lungu, Oscar Nierstrasz, Lukas Renggli and Romain Robbes for the multiple discussions we had and their comments on an early draft of the paper.
%We particular thank Walter Binder for his multiple discussions and advices.
%Our thanks also go to Eliot Miranda for his help on porting \compteur to Cog, the jitted virtual machine of Pharo. 
%We thank Gilad Bracha and Jan Vran\'{y} for the fruitful discussions we had.
%We also thank Andrew P. Black for his precious help on improving the paper.
%We gratefully thank Mar\'ia Jos\'e Cires for her help on the statistical part. 
%We also thank ESUG, the European Smalltalk User Group, for its financial contribution to the presentation of this paper. 
%}
%

\bibliographystyle{plain}
\bibliography{scg}


\end{document}
